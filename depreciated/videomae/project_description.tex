% One-page project description for MS application
\documentclass[11pt]{article}
\usepackage[margin=1in]{geometry}
\usepackage{titlesec}
\usepackage{parskip}
\usepackage{enumitem}
\usepackage[hidelinks]{hyperref}

% Tighten spacing for one-page fit
\setlength{\parskip}{4pt}
\setlength{\parindent}{0pt}
\titlespacing*{\section}{0pt}{6pt}{3pt}
\titlespacing*{\subsection}{0pt}{4pt}{2pt}
\setlist{nosep,leftmargin=*}

\title{Efficient Clip Mining for Autonomous Driving via JEPA-Based Representation Learning}
\author{Applicant Name}
\date{\vspace{-1.0em}}

\begin{document}
\maketitle

\textbf{Objective}\\
Develop a self-supervised pipeline that ranks and selects the most informative driving video clips using a JEPA-style masked video autoencoder (VideoMAE). The goal is to reduce dataset redundancy and training cost while preserving or improving downstream policy performance.

\section*{Motivation}
Modern autonomous driving (AV) datasets contain millions of highly repetitive clips (e.g., normal weather, lane keeping). Uniform sampling wastes compute and slows iteration. Self-supervised predictive models can expose rare or complex scenarios through reconstruction difficulty. By mining such clips, we can train more efficiently, cover edge cases better, and accelerate research cycles.

\section*{Approach}
\textbf{Representation Learning (JEPA/VideoMAE).} We use a JEPA-style masked video autoencoder to encode context and predict masked spatiotemporal patches. The masked-patch reconstruction error serves as a proxy for clip difficulty/novelty.\\
\textbf{Per-Clip Scoring.} For each clip, we compute masked-patch MSE over predicted tokens. Higher scores indicate unusual motion, occlusions, adverse weather, or atypical interactions.\\
\textbf{Selection \\& Diversification.} We sort clips by score and optionally apply embedding-space diversification (e.g., kNN/cluster coverage) to avoid over-representing a single failure mode.\\
\textbf{Evaluation.} We compare models trained on mined subsets vs. uniformly sampled data, measuring policy success (imitation/RL), generalization to domain shifts, and sample efficiency.

\section*{Expected Contributions}
\begin{itemize}
  \item A scalable, label-free clip mining method for AV datasets.
  \item Open, reproducible tooling for per-clip difficulty scoring and visualization.
  \item Evidence that curated subsets achieve comparable or better downstream performance with fewer samples.
\end{itemize}

\section*{Preliminary Results}
Early experiments with VideoMAE (MCG-NJU) produce meaningful score variation and visual reconstructions that highlight masked regions in complex scenes. Score distributions are stable across seeds and correlate with qualitative novelty in sample clips.

\section*{Work Plan (abbrev.)}
\begin{itemize}
  \item Months 1--2: Data ingestion, baseline VideoMAE scoring, visualization, and metrics.
  \item Months 3--4: Diversified selection in embedding space; ablations on score thresholds.
  \item Months 5--6: Downstream evaluation on policy learning tasks; analysis and write-up.
\end{itemize}

\section*{Resources}
Compute with CUDA/MPS; datasets with short driving clips; PyTorch + Hugging Face. Code is modular and runs on CPU for small tests; accelerators recommended for throughput.

\section*{Broader Impact}
Reducing redundant training data lowers energy and compute costs while encouraging robust coverage of rare, safety-critical events—supporting more sustainable and reliable AV development.

\section*{References}
\begin{itemize}
  \item Tong et al., VideoMAE: Masked Autoencoders are Data-Efficient Learners for Self-Supervised Video Pre-Training.
  \item JEPA: Joint Embedding Predictive Architectures for self-supervised learning.
\end{itemize}

\end{document}

